\documentclass[11.5pt,a4paper]{article}

\setlength{\topmargin}{0cm}
\setlength{\headheight}{0.4cm}
\setlength{\headsep}{0.8cm}
\setlength{\footskip}{1cm}
\setlength{\textwidth}{17cm}
\setlength{\textheight}{25cm}
\setlength{\voffset}{-1.5cm}
\setlength{\hoffset}{-0.5cm}
\setlength{\oddsidemargin}{0cm}
\setlength{\evensidemargin}{0cm}

\usepackage{graphicx} % inclusion des figures
\usepackage{amsmath} % collection de symboles math�matiques
\usepackage{amssymb} % collection de symboles math�matiques
\usepackage{subfigure} % 2 Bilder nebeneinander
\usepackage{placeins} %FloatBarrier

%\usepackage[applemac]{inputenc} % utilisation directe des caract�res accentu�s sur mac
%\usepackage[latin1]{inputenc}   % utilisation directe des caract�res accentu�s sur pc
\usepackage[ansinew]{inputenc}	 % klappt unter windows

\usepackage[german,ngerman, english]{babel} %% Deutsch als Hauptsprache
%\usepackage{epstopdf} % wandelt eps-dateien in pdf-datein um
\usepackage[T1]{fontenc} % codage moderne des caract�res sous Latex

\usepackage{tabularx} % Tabellen mit variabler Spaltenbreite

\usepackage{color} % gestion de diff�rentes couleurs
\definecolor{linkcolor}{rgb}{0,0,0.6} % d�finition de la couleur des liens pdf
\usepackage[ pdftex,colorlinks=true, pdfstartview=FitV,
linkcolor= linkcolor, citecolor= linkcolor, urlcolor= linkcolor,
hyperindex=true, hyperfigures=false]{hyperref} % fichiers pdf 'intelligents', avec des liens entre les r�f�rences, etc.

\usepackage{fancyhdr} % Wahl der Kopf- und Fu�zeile
\pagestyle{headings} %zeigt Seitennummer und Abschnittstitel an

\usepackage[center]{caption}


%\pagestyle{fancy} %manuelle Wahl der Kopf- und Fu�zeile
%\fancyhead[L]{\scriptsize \textsc{Electrical properties of a Carbon Nanotube device under extreme conditions}}
%\fancyhead[R]{\scriptsize \textsc{Mario Bomers}}
%\fancyfoot[C]{ \thepage}

\usepackage[normalem]{ulem}
\usepackage{moreverb}
\usepackage{color}
\usepackage{listings}




\begin{document}

\setlength{\parindent}{0pt} %kein Einzug bei neuen Abschnitten

%-------------------------------	Deckblatt		-----------------------------

\thispagestyle{empty}

\includegraphics[height=2cm]{FULogo.jpg}
\hfill %\hfill wird das "a" am linken Seitenrand, "b" am rechten Seitenrand angeordnet ("a" \hfill "b")
\parbox[b]{0.5\textwidth}{{\large Bachelorarbeit Juli 2012\\}}

\vspace{2cm}

\begin{center}

\rule[11pt]{15cm}{0.5pt}

{ \textbf {\Large Large Scale Parallel Simulation of EPR Lineshape Spectra}}

\rule{15cm}{0.5pt}

\vspace{1cm}

\parbox{15cm}{\small
\textbf{Abstract}: This experiment's aim is to measure the Zeeman effect's influence on the spectrum of a mercury light source exposed to magnetic field. Mercury's green $546nm$ and yellow $577nm$  spectral lines split into nine different components each, attributable to the \emph{anomalous Zeeman effect}. Applying a quantum-mechanical theory originally designed for hydrogen-like atoms, we predict the spectral wavelength shifts of the much heavier element mercury and show experimentally the validity of our predictions within a certain tolerance level. }

\vspace{0.5cm}

\end{center}

\vspace{1cm}

\large{

{\bf Group:}  Johannes H\"ormann, Mario Bomers, Finn M\"uller-Hansen}
\vspace{0.3cm}

{\bf Tutor:} Dr. M. Corso

\vspace{1cm}

%\begin{figure}[h!]
%	\centering
%	\includegraphics[width=0.6\textwidth]{screenshot1-546nm.jpg}
%	\caption{Fabry-Perot fringes of 546nm line, no magnetic influence}
%	\label{fringes-546nm}
%\end{figure}

\newpage
\tableofcontents

\vfill
\hfill \today 

%------------------------------------------------------------

\newpage

\section{Basics}
What happens during EPR from a quite general point of view? We excite a certain system with an electromagnetic signal and measure the system's response. In other words, the system $\Phi$ uses the time-resolved input $x(t)$ to generate output $y(t)$:
\begin{equation}
	y(x) = \Phi\{x(t)\}
\end{equation}
If we assume the system to be linear
\begin{equation}
 	\Phi\{\alpha x_1(t) + \beta x_2(t)\} = \alpha y_1(t) + \beta y_2(t)
\end{equation}
and time-invariant
\begin{equation}
	\Phi\{x(t-t_0)\} = y(t-t_0)
\end{equation}
we can expand the input function in a series of some orthonormal basis set $g_k(t)$, or in some integral transform in the continuous limit with basis $g(\tau,t)$ and define the system completely by its set of responses to the basis functions:
\begin{align}
	x(t) & = \sum_k \chi_k g_k(t) = \int_{-\infty}^{\infty} \chi(\tau) g(\tau,t) d\tau \\
	\Rightarrow \Phi{x(t)} & = \sum_k \chi_k \Phi\{g_k(t)\} = \int_{-\infty}^{\infty} \chi(\tau) \Phi\{g(\tau,t)\} d\tau
\end{align}
Using the definition of the Dirac delta function and applying our LTI (linear time-invariant) system
\begin{align}
	x(t) & =  \int_{-\infty}^{\infty} x(\tau) \delta(\tau-t) d\tau \\
	\Rightarrow 	\Phi\{x(t)\} & =  \int_{-\infty}^{\infty} x(\tau) \Phi\{\delta(\tau-t)\} d\tau = \int_{-\infty}^{\infty} x(\tau) h(\tau-t) d\tau = x(t) * h(t)
\end{align}
we find the system's output to be the convolution of the input wit its \emph{pulse response} or \emph{free induction decay (FID)} $h(t) = \Phi\{\delta(t)\}$. 
Furthermore, harmonics are eigenfunctions of LTI systems, and the \emph{frequency response} or \emph{spectrum} $H_(\omega)$ is just the Fourier transform of the system's FID:
\begin{equation}
	\Phi\{e^{i\omega t}\} = \int_{-\infty}^{\infty} e^{i\omega t} h(\tau-t) d\tau =  e^{i\omega t} \int_{-\infty}^{\infty} e^{i\omega (\tau-t)} h(\tau-t) d\tau =  e^{i\omega t} \int_{-\infty}^{\infty} e^{i\omega (\tau)} h(\tau) d\tau = H(\omega) e^{i \omega t}
\end{equation}

\section{$\pi$-pulsed EPR}
The mechanisms of \emph{EPR (Electron Paramegnetic Resonance)}, also called \emph{ESR (Electron Spin Resonance)}, work analogous to the mechanisms of \emph{NMR (Nuclear Magnetic Resonance)}. In diamagnetic materials, all electrons are spin-paired, making the magnetic dipole vanish, and enabling the system to be accessible by NMR, and according to \cite[Chap. 4, p. 107]{nmr-ox}, NMR technique offer two further ``advantages'' in comparison with EPR: First, relaxation time of spin electrons is very short compared with nuclei; second the nuclear spin Hamiltonian offers a broader diversity of interactions giving insight to the system's properties. Nevertheless, EPR experiments are only suitable to investigate systems with unpaired electrons, which exhibit a non-zero electron spin. The basic idea is to perturb a spin system's equilibrium by a small pulsed oscillating magnetic field and record the emitted radiation during the relaxation process. 

When we place a spin sytem inside a static magnetic field $\vec{B}_0 = B_0 \hat{z}$ and let it settle to equilibrium, due to Zeeman effect more electron spins are goint to align parallel to $\vec{B}_0$ than antiparallel, resulting in a net magnetization of the system. The pulse $\vec{B}_1 = B_1 \hat{x} e^{i(\vec{k}\cdot\vec{r}-\omega_1 t} \cdot u(t_0-t)$ linearly polarized in x-direction will tilt the electron spins and disturb the equilibrium in a way we are going to examine in the following:

\subsection{Spin Electron Hamiltonian}
The Hamiltonian of an unpaired spin electron will exhibit several perturbation terms of different nature:
\begin{equation}
 \mathcal{H} = \mathcal{H}_Z + \mathcal{H}_{HF} + \mathcal{H}_D
\end{equation}
\begin{enumerate}
  \item \uline{Zeeman contribution} $\mathcal{H}_Z = - \vec{\mu} \cdot \vec{B}_0 = -\hbar B_0 (\gamma_L \mathbf{L}_z + \gamma_S \mathbf{S}_z)$

  The static magnetic field $\vec{B}_0 = B_0 \hat{z}$ acts a torque on the electron's magnetic dipole moment $\mu$, linearly dependant on its angular momentum and spin. Thus the \emph{Zeeman effect} lifts the spin degeneracy of energy levels, reducing the energy of spins aligned parallel to the magnetic field (-), and increasing the energy of spins aligned antiparallel (+) by the correction term
  \begin{equation}
   E^1\pm = \pm \mu_B \gamma_J B_0 m_J 
  \end{equation}
  resulting from algebraic acrobatics with the Hamiltonian above \cite[insert]{griffiths}
  Boltzmann statistics yields the relation for the equilibrium occupancy of states
  \begin{equation}
    \frac{N_+}{N_-} = e^{- \frac{\Delta E}{k_B T}}
  \end{equation}


  \item \uline{Hyperfine interaction} $\mathcal{H}_{HF} = $
  \item \uline{Magnetic dipole interaction} $\mathcal{H}_{D} = $
\end{enumerate}

\subsection{Rotating Frame}
Suppose we are changing from the lab frame to a frame, which is rotating anticlockwise with angular velocity $\omega_1$ around the z-Axis. The unit vectors of that rotating frame are:
\begin{equation}
 \hat{x}' = \begin{pmatrix} \cos(\omega_1 t) \\ \sin(\omega_1 t) \\ 0 \end{pmatrix} \quad
 \hat{x}' = \begin{pmatrix} - \sin(\omega_1 t) \\  \cos(\omega_1 t)\\ 0 \end{pmatrix} \quad
 \hat{z}' = \hat{z}
\end{equation}
with the conversion 
\begin{align}
 \vec{r}' & = x'\hat{x}' + y'\hat{y}' + z'\hat{z}' \\
	  & = \left[x \cos(\omega_1 t)+ y \sin(\omega_1 t) \right] \hat{x}' + \left[- x \sin(\omega_1 t)+ y \cos(\omega_1 t) \right] + z \hat{z} \\
	  & = R \vec{r} \quad \text{with} \quad R = 
	    \begin{bmatrix} 
	      \cos(\omega_1 t)	& -\sin(\omega_1 t) & 0	\\
	      \sin(\omega_1 t)	& \cos(\omega_1 t) & 0 \\
	      0 & 0 & 1
	    \end{bmatrix}
\end{align}





\section{Spinach}
The Matlab library \emph{Spinach} supplies efficient methods for large-scale spin dynamics simulations. It consists of the \emph{kernel} with the implementation of general spin dynamics simulation techniques and the \emph{user-land} with a collection of different experiements to perform. Basically, the user prepares the description of a spin system, which is then translated by the kernel into the most efficient basis sets, superoperators, etc. The user-land decides how to deal with those objects, whether to apply a pre-established experiment, or whether to perform the kernel's simulation procedures manually. Though Spinach is able to simulate numerous kinds of experiments, in this work we are going to restrict ourselves to standard EPR experiments. For the $\pi$-pulsed EPR, the user-land readily provides the method \verb|pulse_acquire|



\FloatBarrier
\begin{thebibliography}{9}
\bibitem{spinach}
  Dr. Ilya Kuprov,
  \emph{Spin Dynamics, Lecture 1 - 10}.
  University of Oxford,
  2011.

\bibitem{spinach}
  H.J. Hogben, M. Krzystyniak, G.T.P. Charnock, P.J. Hore, Ilya Kuprov,
  \emph{Spinach - A software library for simulation of spin dynamics in large spin systems}.
  Journal of Magnetic Resonance,
  208 (2011) 179-194.

\bibitem{griffiths}
  David J. Griffiths,
  \emph{Introduction to Quantum Mechanics}.
  Pearson, 
  2nd Edition, 
  2005.

\bibitem{nmr-ox}
  Paul T. Callaghan,
  \emph{Translational Dynamics \& Magnetic Resonance. Principles of Pulsed Gradient Spin Echo NMR}.
  Oxford University Press,
  2011.

\end{thebibliography}

\end{document}